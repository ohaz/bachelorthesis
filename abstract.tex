%%%%%%%%%%%%%%%%%%%%%%%%%%%%%%%%%%%%%%%%%%%%%%%%%%%%%%%%%%%%%%%%%%%%%%%%%%%%%%%
%
% Abstract
% Copyright (c) 2010 by tilo.mueller@rwth-aachen.de
% 
%%%%%%%%%%%%%%%%%%%%%%%%%%%%%%%%%%%%%%%%%%%%%%%%%%%%%%%%%%%%%%%%%%%%%%%%%%%%%%%

% Pseudo chapter
\chapter*{\ }


\begin{center}
	\begin{large}
		\textbf{Zusammenfassung}
	\end{large}
\end{center}
\vspace{0.75em}
Google Dienste werden heutzutage von vielen Personen genutzt, teilweise sogar fest in das tägliche Leben integriert. 
Mit den in letzter Zeit häufiger vorkommenden Datenschutzbedenken und dem von Snowden veröffentlichten Material sind genauere Überlegungen zum Datenschutz ein relevantes Thema.

Auf der anderen Seite kann der Nutzer auch eingeschränkt werden, wenn er seine Daten nicht an Google weitergibt. So funktionieren zum Beispiel personalisierte Suchergebnisse nur, wenn die Suchmaschine auch tatsächlich Informationen über den Nutzer hat.

Diese Arbeit soll untersuchen ob es auf Seiten der Nutzer einen Konflikt zwischen dem Angebot an Diensten durch Google und der Privatsphäre gibt. Dazu wurden im Verlauf der Arbeit zwei Forschungsfragen beantwortet: \glqq Wie viel wissen Nutzer von der Datensammlung durch Google?\grqq\ und \glqq Treffen die Nutzer Maßnahmen um gegen dieses Sammeln von Daten vorzugehen?\grqq .

Um diese Fragen zu beantworten wurde ein Fragebogen entworfen und über verschiedene Kanäle verteilt. Nachdem über 800 vollständig ausgefüllte Fragebögen abgegeben wurden wurde die Umfrage beendet und mit der Auswertung begonnen. 

Maßgebliches Ergebnis dieser Arbeit war, dass das typische Paradoxon der Privatsphäre auch hier auftritt - die Nutzer wünschen sich zwar mehr Privatsphäre, unternehmen aber nur selten Aktionen um ihre Privatsphäre zu schützen.

% Pseudo chapter
\chapter*{\ }

\vspace{2em}
\begin{center}
	\begin{large}
		\textbf{Abstract}
	\end{large}
\end{center}
\vspace{0.75em}
Google services nowadays get used by a lot of people, to some extend they even integrate it into their daily life.  Since lately a lot of concerns about privacy have come up and Snowden has leaked information, more detailed thoughts about privacy have become a relevant topic.

On the other hand, if a user does not give Google its data, it may also limit them. Features like personalised search results will only work, if the search engine actually has information about the user.

This thesis will examine if there is a conflict between the offer of features from Google and the privacy when looking at the users. To do that, two research questions were answered in the course of the thesis: \glqq How much do users know of Google collecting their data?\grqq\ and \glqq Do the users take any actions against Google collecting their data?\grqq .

To answer these questions a survey was created and distributed over multiple channels. After over 800 completely filled in results have come in, the survey was stopped and evaluation and analysis started.

The relevant result of this thesis is, that the typical privacy paradox can be seen in this case aswell - the users wish to have better privacy, but thes rarely take action to protect it.