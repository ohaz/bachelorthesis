%%%%%%%%%%%%%%%%%%%%%%%%%%%%%%%%%%%%%%%%%%%%%%%%%%%%%%%%%%%%%%%%%%%%%%%%%%%%%%%
%
% Abstract
% Copyright (c) 2010 by tilo.mueller@rwth-aachen.de
% 
%%%%%%%%%%%%%%%%%%%%%%%%%%%%%%%%%%%%%%%%%%%%%%%%%%%%%%%%%%%%%%%%%%%%%%%%%%%%%%%

% Pseudo chapter
\chapter*{\ }


\begin{center}
	\begin{large}
		\textbf{Zusammenfassung}
	\end{large}
\end{center}
\vspace{0.75em}
Google Dienste werden heutzutage von vielen Personen genutzt, teilweise sogar fest in das tägliche Leben integriert. 
Mit den in letzter Zeit häufiger werdenden Datenschutzbedenken und dem von Snowden veröffentlichten Material sollte man sich genaue Überlegungen zum Datenschutz machen. 

Auf der anderen Seite kann der Nutzer auch eingeschränkt werden, wenn er seine Daten nicht an Google weitergibt. So funktionieren zum Beispiel personalisierte Suchergebnisse nur, wenn die Suchmaschine auch tatsächlich Informationen über den Nutzer hat.

Diese Arbeit soll untersuchen ob es auf Seiten der Nutzer einen Konflikt zwischen dem Angebot an Diensten durch Google und der Privatsphäre gibt. Dazu wurden im Verlauf der Arbeit zwei Forschungsfragen beantwortet: \glqq Wie viel wissen Nutzer von der Datensammlung durch Google?\grqq\ und \glqq Treffen sie Maßnahmen um gegen dieses Sammeln von Daten vorzugehen?\grqq .

Um diese Fragen zu beantworten wurde ein Fragebogen mit 39 Fragen entworfen und über verschiedene Methoden verteilt. Nachdem über 800 vollständig ausgefüllte Fragebögen abgegeben wurden wurde die Umfrage beendet und mit der Auswertung begonnen. Dabei wurden Hypothesen falls möglich belegt und danach die Forschungsfragen beantwortet. 

Dabei kam heraus, dass das typische Paradoxon der Privatsphäre auch hier auftritt - die Nutzer wünschen sich zwar mehr Privatsphäre, unternehmen aber nur selten Aktionen um ihre Privatsphäre zu schützen.

Am Ende werden noch einige Ideen für zukünftige Arbeiten in diesem Themenbereich genannt, unter anderem eine Umfrage bei der auf eine homogenere Aufteilung der Bevölkerungsgruppen geachtet werden sollte.

\vspace{2em}
\begin{center}
	\begin{large}
		\textbf{Abstract}
	\end{large}
\end{center}
\vspace{0.75em}
In dieser Arbeit wurde der Konflikt zwischen Privatsphäre und der Verwendung von Google Diensten untersucht. Mithilfe einer Umfrage wurden Forschungsfragen beantwortet. Die Frage \glqq Wie viel wissen Nutzer von der Datensammlung durch Google?\grqq\ konnte damit beantwortet werden, dass bei einem Großteil der Teilnehmer Wissen vorhanden ist. Die zweite Forschungsfrage \glqq Treffen sie Maßnahmen um gegen dieses Sammeln von Daten vorzugehen?\grqq\ wurde bis auf wenige Ausnahmen negativ beantwortet. Dieser Konflikt ist mit dem \glqq Privatsphäre Paradox\grqq\ zu erklären.