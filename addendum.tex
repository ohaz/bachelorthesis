%%%%%%%%%%%%%%%%%%%%%%%%%%%%%%%%%%%%%%%%%%%%%%%%%%%%%%%%%%%%%%%%%%%%%%%%%%%%%%%
%
% Anhang
% Copyright (c) 2010 by tilo.mueller@rwth-aachen.de
% 
%%%%%%%%%%%%%%%%%%%%%%%%%%%%%%%%%%%%%%%%%%%%%%%%%%%%%%%%%%%%%%%%%%%%%%%%%%%%%%%

\chapter{Anhang}

%Section		Depth
%\part			1
%\chapter		2
%\section 		3
%\subsection 		4
%\subsubsection 	5 	-> from here on it's not in default TOC anymore
%\paragraph	 	6
%\subparagraph 		7


\begin{table}
	\begin{tabular}[]{ p{1.5cm} | p{1.5cm} | p{5cm} | p{5cm} }
	\hline
	Definition ID & Frage ID & Frage & Antwortmöglichkeiten \\
	\hline \hline
	0 & 1 & Welche der folgenden Begriffe würden Sie dem Unternehmen Google zuordnen? & Checkbox: Android(Smartphone Betriebssystem), Youtube (Videoplatform), Gmail(Email Service), Picasa (Bildverarbeitung), Facebook (soziales Netzwerk), Google+ (soziales Netzwerk), WhatsApp (Instant Messenger), Hangouts (Instant Messenger), Windows (Betriebssystem), World of Warcraft (Videospiel), Amazon (Versandhändler), Firefox (Internetbrowser), Chrome (Internetbrowser)\\
	\hline
	0 & 2 & Wie aktiv nutzen Sie die folgenden Google Dienste? & 1-5 (1 = gar nicht, 5 = sehr intensiv): Google Suche, Google Mail, Android, GOogle Chrome, Youtube, Google+, Google Maps, Picasa, Hangouts, Google Docs, Google Kalender\\
	\hline
	0 & 3 & Wie viele Google Accounts haben Sie? & 0, 1, 2, 2+\\
	\hline
	0 & 4 & Was ist der Grund dafür, dass Sie mehrere Accounts haben? & Textfeld\\
	\hline
	0 & 5 & Wie wichtig sind Ihnen die folgenden Google Dienste? & 1-5 + nutze ich nicht: Google Suche, Google Mail, Android, Google Chrome, Youtube, Google Maps, Google+, Picasa, Hangouts, Google Docs, Google Kalender\\
	\hline
	2 & 6 & Gab es in der Vergangenheit ein Ereignis das ihr Vertrauen in Google stark beeinflusst hat? & Ja (Textfeld), Nein, Weiß nicht\\
	\hline
	1 & 7 & Bietet Google auf Nutzer zugeschnittene Werbung an? & Ja, Nein, Weiß nicht\\
	\hline
	1 & 8 & Bietet Google auf Nutzer zugeschnittene Suchergebnisse an? & Ja, Nein, Weiß nicht\\
	\hline
	4 & 9 & Ich denke, Google weiß folgendes über micht: & Checkbox: Name, Wohnort, Aufenthaltsort, Telefonnummer, Freundeskreis, Beziehungsstatus, politische Gesinnung, Religion, Krankheiten, derzeitige Stimmungslage, Aussehen, Beruf, Interessen, Sonstiges (Textfeld)\\
	\hline
	
	\end{tabular}
	\caption{Fragen}\label{addendumquestions}
\end{table}

\begin{table}
	\begin{tabular}[]{ p{1.5cm} | p{1.5cm} | p{5cm} | p{5cm} }
	\hline
	Definition ID & Frage ID & Frage & Antwortmöglichkeiten \\
	\hline \hline
	2 & 10 & Meine Daten sind bei Google sicher. & 1 (gar nicht) - 5 (sehr)\\
	\hline
	2 & 11 & Ich vertraue darauf, dass diese Firma meine Privatsphäre schützt: & 1 (gar nicht) - 5 (sehr): Microsoft, Facebook, Apple, Google, Paypal, Amazon\\
	\hline
	5 / 2 & 12 & Kennen Sie alternative Suchmaschinen zur Google Suche? & Ja (Textbox), Nein (Textbox), Weiß nicht\\
	\hline
	0 & 13 & Nutzen Sie dennoch die Google Suche? Warum? & Ja, Nein, Weiß nicht + Begründung Textbox\\
	\hline
	1 & 14 & Mit welchen großen Firmen bzw. Organisationen tauscht Google Ihrer Meinung nach im Allgemeinen Daten über die Nutzer aus? & Microsoft, Apple, Facebook, Paypal, Amazon, staatliche Einrichtungen, Keine, Sonstiges (Textbox)\\
	\hline
	3 & 15 & Sehen Sie das als Nutzen von personenbezogenen Daten seitens Google als Vorteil für Sie? & Ja, Nein, Weiß nicht\\
	\hline
	3 & 16 & Welche Vorteile sind das? & Textbox\\
	\hline
	3 & 17 & Fühlen Sie sich bei der Nutzung von Google Diensten gegenüber Google anonym? & Ja, Nein, Weiß nicht\\
	\hline
	4 & 18 & Würden Sie gegenüber Google gerne anonym sein? & Ja, Nein, Weiß nicht\\
	\hline
	5 & 19 & Haben Sie Maßnahmen unternommen um anonym gegenüber Google zu sein? Wenn ja, welche? & Ja (Textbox), Nein (Textbox), Weiß nicht\\
	\hline
	1 & 20 & Bietet Google die Möglichkeit den eigenen Account zu löschen? & Ja, Nein, Weiß nicht\\
	\hline
	5 & 21 & Würden Sie Ihren Google Account löschen um an Privatsphäre zu gewinnen? & Ja, Nein, Weiß nicht + Textbox\\
	\hline
	5 & 22 & Bei welchen Google Diensten haben Sie bereits Privatsphäre Einstellungen geändert? & Google Suche, Google Mail, Android, Google Chrome, Youtube, Google Maps, Google+, Picasa, Hangouts, Google Docs, Google Kalender, Sonstiges\\
	\hline
	2 & 23 & Wie stark vertrauen Sie Google? & 1 (gar nicht) - 5 (sehr)\\
	\hline
	3 & 24 & Google bietet mir ausreichend Möglichkeiten meine Privatsphäre zu schützen. & 1 (gar nicht) - 5 (stark)\\
	\hline
	
	\end{tabular}
	\caption{Fragen (Fortsetzung)}
\end{table}

\begin{table}
	\begin{tabular}[]{ p{1.5cm} | p{1.5cm} | p{5cm} | p{5cm} }
	\hline
	Definition ID & Frage ID & Frage & Antwortmöglichkeiten \\
	\hline \hline
	2 & 25 & Darf Google Fotos zusammen mit Namen zu Werbezwecken verwenden? & Ja, Nein, Weiß nicht\\
	\hline
	2 & 26 & Gibt es eine Einstellung, die es Google verbietet Ihr Foto und Ihren Namen zu Werbezwecken zu verwenden? & Ja, Nein, Weiß nicht\\
	\hline
	5 & 27 & Haben Sie diese Einstellung aktiv? & Ja, Nein, Weiß nicht\\
	\hline
	4 & 28 & Wie viele Informationen über Ihr privates Leben geben Sie im Internet heraus? & 1 (viel weniger als im privaten Leben) - 5 (viel mehr als im realen Leben)\\
	\hline
	4 & 29 & Durch die Nutzung von Google Diensten gebe ich zu viel meiner Privatsphäre auf & 1 (gar nicht) - 5 (voll und ganz einverstanden)\\
	\hline
	Demo1 & & Sind Sie & Männlich, Weiblich\\
	\hline
	Demo2 & & Wie alt sind sie? & Freitextfeld (Integer)\\
	\hline
	Demo3 & & Welchen höchsten Schulabschluss haben Sie? & Kein Abschluss, Hauptschulabschluss, Realschulabschluss, Abitur/Fachabitur, Bachelor/Master/Diplom, Promotion\\
	\hline
	Demo4 & & Sind Sie derzeit & In Ausbildung, Student/-in, Berufstätig, In Rente, Nicht Berufstätig\\
	\hline
	Demo5 & & Wie hoch sind Ihre Informatik Kenntnisse? & 1 (keine) - 5 (hohe)\\
	\hline
	Demo6 & & Haben Sie Kenntnisse im Bereich IT-Sicherheit? & 1 (keine) - 5 (hohe)\\
	\hline
	Demo7 & & Haben Sie einen Bildungsabschluss in der Informatik oder einem anderen IT-nahem Fachbereich? & Ja, Nein, Weiß nicht\\
	\hline
	& & Falls Sie Feedback zu dieser Umfrage haben können Sie dieses hier eintragen:& Textbox\\
	\hline	
	\end{tabular}
	\caption{Fragen (Fortsetzung)}
\end{table}