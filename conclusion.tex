%%%%%%%%%%%%%%%%%%%%%%%%%%%%%%%%%%%%%%%%%%%%%%%%%%%%%%%%%%%%%%%%%%%%%%%%%%%%%%%
%
% conclusion
% Copyright (c) 2010 by tilo.mueller@rwth-aachen.de
% 
%%%%%%%%%%%%%%%%%%%%%%%%%%%%%%%%%%%%%%%%%%%%%%%%%%%%%%%%%%%%%%%%%%%%%%%%%%%%%%%

\chapter{Fazit}

%Section		Depth
%\part			1
%\chapter		2
%\section 		3
%\subsection 		4
%\subsubsection 	5 	-> from here on it's not in default TOC anymore
%\paragraph	 	6
%\subparagraph 		7

\section{Zusammenfassung}
Aus den 856 ausgewerteten Fragebögen ergaben sich einige interessante Ergebnisse. Allerdings ist relevant, dass mit 71,6\% die Teilnehmerzahl überwiegend männlich war und über 90\% der Teilnehmer Studenten oder Personen mit Abitur waren. Diese beiden Fakten sorgen dafür, dass die Umfrage nicht repräsentativ für die Allgemeinheit ist.

Es lässt sich sagen, dass Google von den Nutzern sehr aktiv genutzt wird. Fast 90\% der Teilnehmer haben angegeben dass sie Google oft bzw. sehr oft nutzen.

Des weiteren kann man sagen, dass das Wissen der Nutzer über Google im in großen Teilen gut ist. Ein Großteil der Nutzer weiß zum Beispiel, dass Google personalisierte Suchergebnisse anbietet. Es gibt aber Lücken im Wissen, nur ein knappes Drittel der Teilnehmer weiß, dass man seinen Google Account löschen kann.

Das Vertrauen in Google ist eher gering - ein Drittel der Umfrageteilnehmer meint, dass ihre Daten bei Google nicht sicher sind.

Die Nutzer nehmen ein Risiko für ihre Privatsphäre wahr, so fühlen sich 85,9\% der Teilnehmer gegenüber Google nicht anonym.

In großen Teilen sind sich Nutzer über das Aufgeben ihrer Privatsphäre bewusst - viele Teilnehmer glauben zu wissen, dass Google viele ihrer privaten Daten, wie zum Beispiel Telefonnummer oder Wohnort kennt.

Tabelle \ref{hypothesenangenommen} zeigt, welche Hypothesen mit statistischer Signifikanz angenommen werden konnten und welche nicht belegt werden konnten.

Zu den Forschungsfragen lässt sich sagen, dass Nutzer wissen, dass Google ihre Daten sammelt, sie aber nichts oder nur wenig dagegen unternehmen. Hier zeigt sich wieder das typische Privatsphäre Paradoxon auf, welches im Bezug auf den Datenschutz ein bekanntes Problem ist.

\begin{table}
	\begin{tabular}[]{l | c }
	Hypothese & Angenommen\\\hline\hline
	Je aktiver eine Person Google nutzt,\\ desto mehr Kenntnisse über Googles Datenschutz bekommt sie. & Nein\\ \hline
	Je mehr eine Person über Googles Datenschutz weiß,\\ desto bewusster ist ihr das Aufgeben der Privatsphäre. & Ja\\ \hline
	Je höher das Vertrauen in Google ist,\\ desto geringer wird das Risiko eingeschätzt.&Ja\\\hline
	Je höher das Risiko eingeschätzt wird,\\ desto mehr Schutzmaßnahmen werden unternommen.&Nein\\\hline
	Je mehr Schutzmaßnahmen eine Person verwendet,\\ desto bewusster ist ihr das Aufgeben der Privatsphäre.&Nein\\\hline
	\end{tabular}
	\caption{Welche Hypothesen wurden statisch signifikant angenommen}\label{hypothesenangenommen}
\end{table}

\section{Ausblick}
In dem in dieser Arbeit behandelten Thema gibt es noch weite Bereiche zum weiteren Forschen.

Wie schon bei der Behandlung der demographischen Auswertung (\ref{sec:demo}) genannt, waren ein Großteil der Teilnehmer dieser Umfrage Studenten oder Personen, die ein Abitur als Abschluss hatten. Die Bevölkerungsgruppe, die als höchsten Bildungsabschluss einen Realschul- oder einen Hauptschulabschluss haben wurden kaum erreicht. Eine weitere Forschungsarbeit könnte in die Richtung gehen, dies weiter zu erforschen und sich z.B. im Besonderen auf diese Gruppen zu spezialisieren.

Des weiteren wäre es sinnvoll, Studien durchzuführen, bei denen das Wissen und die Taten der Nutzer direkt getestet werden. So wäre es zum Beispiel praktisch, Freiwillige zu fragen, ob man Einblick in die Sicherheitseinstellungen ihres Smartphones erhalten darf, um diese zu untersuchen. Alternativ könnte man Versuchsteilnehmer bitten, unter Aufsicht Sicherheitseinstellungen auf Google Seiten vorzunehmen.

Somit könnte direkt geprüft werden, wie viel die Teilnehmer tatsächlich eingestellt haben, oder ob ihre Aussagen dadurch beeinflusst sind, dass sie selbst gar nicht alle problematischen Bereiche kennen. Zum Beispiel sei hier angeführt, dass vielen Leuten eventuell gar nicht bewusst ist, dass allein dadurch, dass sie sich im selben WLAN befinden, oder einen ähnlichen GPS-Ort haben, bereits eine Verbindung zwischen ihnen und anderen aufgestellt werden kann. Fragen dieser Art wären eine weitere Untersuchung wert.

Weiterhin wäre es interessant zu erfahren, wie man das in Sektion \ref{sec:forschungsfragen} beschriebene Paradoxon der Privatsphäre lösen kann, beziehungsweise ob man es überhaupt lösen kann.


\chapter{Danksagung}
Mein Dank gilt Nadina Hintz und Zinaida Benenson, welche mich durch die Bachelorarbeit hindurch mit hilfreichen Vorschlägen begleitet haben.

Des weiteren danke ich natürlich meiner Familie, die mir während der Bachelorarbeit immer mit psychologischem Beistand beiseite gestanden sind und mir durch die stressige Zeit hindurch geholfen hat.

Zu guter Letzt danke ich den vielen Teilnehmern an der Umfrage dafür, dass sie Zeit aufgewandt haben um mir bei der Ausarbeitung dieser Arbeit zu helfen.