%%%%%%%%%%%%%%%%%%%%%%%%%%%%%%%%%%%%%%%%%%%%%%%%%%%%%%%%%%%%%%%%%%%%%%%%%%%%%%%
%
% conclusion
% Copyright (c) 2010 by tilo.mueller@rwth-aachen.de
% 
%%%%%%%%%%%%%%%%%%%%%%%%%%%%%%%%%%%%%%%%%%%%%%%%%%%%%%%%%%%%%%%%%%%%%%%%%%%%%%%

\chapter{Fazit}

%Section		Depth
%\part			1
%\chapter		2
%\section 		3
%\subsection 		4
%\subsubsection 	5 	-> from here on it's not in default TOC anymore
%\paragraph	 	6
%\subparagraph 		7

\section{Zusammenfassung}

\section{Ausblick}
In dem in dieser Arbeit behandelten Bereich steht noch viel Bereich zum weiteren Forschen an. Wie schon bei der Behandlung der demographischen Auswertung (\ref{sec:demo}) genannt, waren ein Großteil der Teilnehmer dieser Umfrage Studierte oder Personen die ein Abitur als Abschluss hatten - die Bevölkerungsgruppe, die als höchsten Bildungsabschluss einen Realschul- oder einen Hauptschulabschluss haben wurden kaum erreicht. Eine weitere Forschungsarbeit könnte in die Richtung gehen dies weiter zu erforschen und sich z.B. im Besonderen auf diese Gruppen zu spezialisieren.\\
Des weiteren wäre es sinnvoll, Studien durchzuführen bei denen das Wissen und die Taten der Nutzer direkt getestet werden. So wäre es zum Beispiel sinnvoll, Freiwillige zu fragen ob man sich kurz ihr Android Smartphone ausleihen darf um dort die Sicherheitseinstellungen zu prüfen, oder Versuchsteilnehmern unter Aufsicht Sicherheitseinstellungen auf Google Seiten vornehmen zu lassen. Somit könnte direkt geprüft werden wie viel die Teilnehmer tatsächlich eingestellt haben oder ob die Aussagen die sie machen dadurch beeinflusst sind, dass sie selbst gar nicht alle problematischen Bereiche kennen.\\

