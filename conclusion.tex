%%%%%%%%%%%%%%%%%%%%%%%%%%%%%%%%%%%%%%%%%%%%%%%%%%%%%%%%%%%%%%%%%%%%%%%%%%%%%%%
%
% conclusion
% Copyright (c) 2010 by tilo.mueller@rwth-aachen.de
% 
%%%%%%%%%%%%%%%%%%%%%%%%%%%%%%%%%%%%%%%%%%%%%%%%%%%%%%%%%%%%%%%%%%%%%%%%%%%%%%%

\chapter{Fazit}

%Section		Depth
%\part			1
%\chapter		2
%\section 		3
%\subsection 		4
%\subsubsection 	5 	-> from here on it's not in default TOC anymore
%\paragraph	 	6
%\subparagraph 		7

\section{Zusammenfassung}

Google Dienste werden heutzutage von vielen Personen genutzt, teilweise sogar fest in das tägliche Leben integriert. 
Mit den in letzter Zeit häufiger werdenden Datenschutzbedenken und dem von Snowden veröffentlichten Material sollte man sich genaue Überlegungen zum Datenschutz machen. 

Auf der anderen Seite kann der Nutzer auch eingeschränkt werden, wenn er seine Daten nicht an Google weitergibt. So funktionieren zum Beispiel personalisierte Suchergebnisse nur, wenn die Suchmaschine auch tatsächlich Informationen über den Nutzer hat.

Diese Arbeit soll untersuchen ob es auf Seiten der Nutzer einen Konflikt zwischen dem Angebot an Diensten durch Google und der Privatsphäre gibt. Dazu wurden im Verlauf der Arbeit zwei Forschungsfragen beantwortet: \glqq Wie viel wissen Nutzer von der Datensammlung durch Google?\grqq\ und \glqq Treffen sie Maßnahmen um gegen dieses Sammeln von Daten vorzugehen?\grqq .

Um diese Fragen zu beantworten wurde ein Fragebogen mit 39 Fragen entworfen und über verschiedene Methoden verteilt. Nachdem über 800 vollständig ausgefüllte Fragebögen abgegeben wurden wurde die Umfrage beendet und mit der Auswertung begonnen. Dabei wurden Hypothesen falls möglich belegt und danach die Forschungsfragen beantwortet. 

Dabei kam heraus, dass das typische Paradoxon der Privatsphäre auch hier auftritt - die Nutzer wünschen sich zwar mehr Privatsphäre, unternehmen aber nur selten Aktionen um ihre Privatsphäre zu schützen.

Am Ende werden noch einige Ideen für zukünftige Arbeiten in diesem Themenbereich genannt, unter anderem eine Umfrage bei der auf eine homogenere Aufteilung der Bevölkerungsgruppen geachtet werden sollte.

\section{Ausblick}
In dem in dieser Arbeit behandelten Thema gibt es noch weite Bereiche zum weiteren Forschen.

Wie schon bei der Behandlung der demographischen Auswertung (\ref{sec:demo}) genannt, waren ein Großteil der Teilnehmer dieser Umfrage Studierte oder Personen die ein Abitur als Abschluss hatten. Die Bevölkerungsgruppe, die als höchsten Bildungsabschluss einen Realschul- oder einen Hauptschulabschluss haben wurden kaum erreicht. Eine weitere Forschungsarbeit könnte in die Richtung gehen, dies weiter zu erforschen und sich z.B. im Besonderen auf diese Gruppen zu spezialisieren.

Des weiteren wäre es sinnvoll, Studien durchzuführen bei denen das Wissen und die Taten der Nutzer direkt getestet werden. So wäre es zum Beispiel sinnvoll, Freiwillige zu fragen, ob man sich kurz ihr Android Smartphone ausleihen darf, um dort die Sicherheitseinstellungen zu prüfen. Alternativ könnte man Versuchsteilnehmer bitten unter Aufsicht Sicherheitseinstellungen auf Google Seiten vornehmen zu lassen.

Somit könnte direkt geprüft werden wie viel die Teilnehmer tatsächlich eingestellt haben oder ob die Aussagen die sie machen dadurch beeinflusst sind, dass sie selbst gar nicht alle problematischen Bereiche kennen. Zum Beispiel sei hier angeführt, dass vielen Leuten eventuell gar nicht bewusst ist, dass allein dadurch, dass sie sich im selben WLAN befinden, oder einen ähnlichen GPS-Ort haben, bereits eine Verbindung zwischen ihnen und anderen aufgestellt werden kann. Fragen dieser Art wären eine weitere Untersuchung wert.

Weiterhin wäre es interessant zu erfahren, wie man das in Sektion \ref{sec:forschungsfragen} beschriebene Paradoxon der Privatsphäre lösen kann, beziehungsweise ob man es überhaupt lösen kann.
