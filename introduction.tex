%%%%%%%%%%%%%%%%%%%%%%%%%%%%%%%%%%%%%%%%%%%%%%%%%%%%%%%%%%%%%%%%%%%%%%%%%%%%%%%
%
% Introduction
% Copyright (c) 2010 by tilo.mueller@rwth-aachen.de
% 
%%%%%%%%%%%%%%%%%%%%%%%%%%%%%%%%%%%%%%%%%%%%%%%%%%%%%%%%%%%%%%%%%%%%%%%%%%%%%%%

\chapter{Einführung}

%Section		Depth
%\part			1
%\chapter		2
%\section 		3
%\subsection 		4
%\subsubsection 	5 	-> from here on it's not in default TOC anymore
%\paragraph	 	6
%\subparagraph 		7

\section{Motivation}
Mit den vor kurzem aufgetretenen Veröffentlichungen durch Edward Snowden sind der Datenschutz und die Privatsphäre wieder zu einem relevanten Gesprächsthema geworden. So sagt zum Beispiel Hanspeter Thür, der Datenschutzbeauftragte der Schweiz: \glqq Privatsphäre wird zu einem Privileg\grqq \cite{nzzdatenschutzprivileg}.

Durch diese Debatten geraten auch große Unternehmen im Internet wieder in den Fokus. Da auch in Deutschland die Google Inc. eine sehr große Nutzerbasis hat (alleine in Deutschland nutzen ungefähr 38 Millionen Nutzer die Suchmaschine von Google \cite{statistagoogle}), gerät Google Inc. des öfteren mit Datenschützern in Konflikt. Als Beispiele lassen sich  "Datenschützer: Google verstößt gegen geltendes Recht" \cite{gulligooglegeltendesrecht} und \glqq Geodaten im Spannungsfeld zwischen Datenschutz und Informationsfreiheit \cite{geodata} nennen. Wie in den Beispielen verdeutlicht ist die Google Suchmaschine dabei nicht der einzige Google Dienst der in den Mittelpunkt von Diskussionen gerät.

Spätestens seit Julian Assange Google eine "privatisierte NSA" genannt hat \cite{assangegooglensa}, sollte man sich im Bezug auf persönliche Daten bei Google besonders überlegen, ob es das Verwenden von Google wert ist.

Es gibt also einen potentiellen Konflikt zwischen der Nutzung von Google Diensten und dem Datenschutz. Dieser soll im Rahmen dieser Arbeit untersucht werden.

\section{Zielsetzung und Forschungsfragen}
Diese Arbeit soll die Nutzer von Google näher betrachten und herausfinden wie diese Nutzer über ihren Datenschutz im Bezug auf Google denken und wie sie vorgehen um ihre Daten zu schützen. Hierzu wird eine Umfrage durchgeführt die Nutzer über ihr Verhalten auf Google und ihre Einstellung gegenüber Google befragt.

Die grundsätzlichen Forschungsfragen die behandelt werden sind
\begin{enumerate}
\item Wie viel wissen Nutzer von der Datensammlung durch Google?
\item Treffen sie Maßnahmen um gegen dieses Sammeln von Daten vorzugehen?
\end{enumerate}