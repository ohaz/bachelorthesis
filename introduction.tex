%%%%%%%%%%%%%%%%%%%%%%%%%%%%%%%%%%%%%%%%%%%%%%%%%%%%%%%%%%%%%%%%%%%%%%%%%%%%%%%
%
% Introduction
% Copyright (c) 2010 by tilo.mueller@rwth-aachen.de
% 
%%%%%%%%%%%%%%%%%%%%%%%%%%%%%%%%%%%%%%%%%%%%%%%%%%%%%%%%%%%%%%%%%%%%%%%%%%%%%%%

\chapter{Einführung}

%Section		Depth
%\part			1
%\chapter		2
%\section 		3
%\subsection 		4
%\subsubsection 	5 	-> from here on it's not in default TOC anymore
%\paragraph	 	6
%\subparagraph 		7

\section{Motivation}
Mit den vor kurzem aufgetretenen Veröffentlichungen durch Edward Snowden sind der Datenschutz und die Privatsphäre wieder zu einem relevanten Gesprächsthema geworden. So sagt zum Beispiel Hanspeter Thür, der Datenschutzbeauftragte der Schweiz: \glqq Privatsphäre wird zu einem Privileg\grqq\ \cite{nzzdatenschutzprivileg}.

Durch diese Debatten geraten auch große Unternehmen im Internet wieder in den Fokus. Auch in Deutschland hat die Google Inc. eine sehr große Nutzerbasis (alleine in Deutschland nutzen ungefähr 38 Millionen Nutzer die Suchmaschine von Google \cite{statistagoogle}, Android hat einen Marktanteil am Absatz von 70,1\% \cite{statistaandroid}). Die große Nutzerbasis legt es nahe, dass die Google Inc. auch in Deutschland des öfteren mit Datenschützern in Konflikt gerät. Als Beispiel lässt sich \citet{geodata} nennen. Wie in dem Beispiel verdeutlicht ist die Google Suchmaschine dabei nicht der einzige Google Dienst der in den Mittelpunkt von Diskussionen gerät.

In seinem Buch über Google und Wikileaks schrieb Julian Assange, dass Google um die 2 Millionen USD von der NSA angenommen hat, um die NSA mit Suchprogrammen auszurüsten \citet{assangebook}. Spätestens nachdem Wahrheiten wie diese an die Öffentlichkeit geraten, sollte man sich im Bezug auf persönliche Daten besonders überlegen, ob man diese Google überlassen will.

Im selben Buch wird beschrieben, dass Google im Jahre 2010 einen Vertrag mit der NSA eingegangen ist, in dem es offiziell darum ging, dass die NSA Sicherheitslücken in Google Diensten aufspüren soll. In den wenigen Informationen, die man über diesen Vertrag kennt, ist enthalten, dass es sich um einen Vertrag zum \glqq Austausch von Informationen\grqq\ handelt.

Es gibt also einen potentiellen Konflikt zwischen der Nutzung von Google Diensten und dem Datenschutz. Dieser soll im Rahmen dieser Arbeit untersucht werden.

\section{Zielsetzung und Forschungsfragen}
\label{sec:questions}
Diese Arbeit soll die Hintergründe der Nutzer von Google näher betrachten und herausfinden wie diese Nutzer über ihren Datenschutz im Bezug auf Google denken und wie sie vorgehen um ihre Daten zu schützen. Hierzu wird eine Umfrage durchgeführt, in der die Nutzer über ihr Verhalten auf Google und ihre Einstellung gegenüber Google befragt werden.

Die grundsätzlichen Forschungsfragen die behandelt werden sind
\begin{enumerate}
\item Wie viel wissen Nutzer von der Datensammlung durch Google?
\item Treffen sie Maßnahmen um gegen dieses Sammeln von Daten vorzugehen?
\end{enumerate}