%%%%%%%%%%%%%%%%%%%%%%%%%%%%%%%%%%%%%%%%%%%%%%%%%%%%%%%%%%%%%%%%%%%%%%%%%%%%%%%
%
% methodology
% Copyright (c) 2010 by tilo.mueller@rwth-aachen.de
% 
%%%%%%%%%%%%%%%%%%%%%%%%%%%%%%%%%%%%%%%%%%%%%%%%%%%%%%%%%%%%%%%%%%%%%%%%%%%%%%%

\chapter{Methodik}

Die Studie wurde als Online-Umfrage entwickelt, mit dem direkten Ziel die Forschungsfragen zu beantworten. Dazu wurden die Forschungsfragen und die zugehörigen Hypothesen aufgestellt und mit diesen Mitteln der Fragebogen erstellt.

\section{Fragebogenkonstruktion}
Um die Umfrage zu erstellen wurde zuerst ein umfassender Fragenkatalog zum Thema Google erstellt. Dabei wurde noch nicht direkt auf die Hypothesen eingegangen und die Fragen deckten teilweise zu große oder zu kleine Bereiche ab. Aus diesem Fragenkatalog wurden danach die sinnvollen Fragen herausgefiltert und die restlichen verworfen. Die sinnvollen Fragen wurden überarbeitet und in Fragen umgewandelt die für eine Umfrage passender sind. Dieser neue Fragenkatalog wurde erweitert und belief sich am Ende auf ungefähr 60 Fragen.
Diese Fragen wurden einer neuen Filterung unterzogen bei der diesmal vor allem die Relevanz für die Hypothesen im Vordergrund stand. Des weiteren wurden einige Fragen ausgefiltert die in einer Online-Umfrage nicht verwendbar gewesen wären, da für diese Fragen eine direkte Interaktion mit dem Benutzer nötig gewesen wäre (z.b. "Gehen Sie auf Google und geben sie den Begriff 'Restaurant' ein - was fällt ihnen dabei auf?", diese Frage hätte eventuell bei vielen Benutzern zum Abbruch der Umfrage geführt). Das Ergebniss war eine Umfrage mit 39 Fragen, die in Kategorien unterteilt waren, die ein schnelles und zielstrebiges durcharbeiten des Fragebogens vereinfachen sollte. Diese Kategorien sind nicht an spezielle Themen gebunden sondern eher so unterteilt, dass Fragen die eventuell die Meinung beeinflussen könnten, wie zum Beispiel "Durch die Nutzung von Google Diensten gebe ich zu viel von meiner Privatsphäre auf" erst gegen Ende der Umfrage kommen, wohingegen am Anfang eher allgemeine Wissensfragen stehen.
Der letzte Teil der Umfrage ist eine Liste an demografischen Fragen die eine Einordnung der Teilnehmer erleichtern soll. Am Ende dieses Teiles steht noch eine Frage nach Feedback zur Umfrage.
Der Fragebogen war auf eine ungefähre Dauer von 10-15 Minuten ausgelegt. Nachdem dies bei den Testteilnehmern dem tatsächlichem Zeitverbrauch entsprochen hat (10min 58s, 13min 13s, 14min 43s, 11min 52s) wurde die Länge des Fragebogens nicht mehr verändert.
\begin{figure}[H]
\centering
\includegraphics[height=\textheight]{images/umldia}\\
\caption{Anordnung der Fragen}\label{umldia}
\end{figure}

\section{Rekrutierung der Teilnehmer}
Die Umfrage wurde zuerst über Facebook und über Familienmitglieder und Bekannte verbreitet. Nachdem so schon eine relativ große Anzahl an Antworten bekommen wurden wurde die Umfrage noch über den Email-Verteiler der Technischen Fakultät der Universität Erlangen-Nürnberg verbreitet. Insgesamt wurden somit innerhalb weniger Tage 856 vollständige Antworten erhalten, was zum vorzeitigen Abbruch führte. Eine zu große Anzahl an Antworten hätte die Auswertung nur gebremst und mit über 850 sollte bereits eine gute Auswertung möglich sein.