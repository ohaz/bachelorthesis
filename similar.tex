%%%%%%%%%%%%%%%%%%%%%%%%%%%%%%%%%%%%%%%%%%%%%%%%%%%%%%%%%%%%%%%%%%%%%%%%%%%%%%%
% 
% Prerequisites 
% Copyright (c) 2010 by tilo.mueller@rwth-aachen.de
% 
%%%%%%%%%%%%%%%%%%%%%%%%%%%%%%%%%%%%%%%%%%%%%%%%%%%%%%%%%%%%%%%%%%%%%%%%%%%%%%%

\chapter{Ähnliche Arbeiten}
Die Problematik der Privatsphäre bei der Kommunikation mit Google Diensten ist ein Thema zu der es bereits einige Artikel gibt. So beschreibt zum Beispiel Omer Tene in dem Artikel "What Google knows: Privacy and internet search engines" \cite{tene2007google}, dass die größten Probleme der Privatsphäre sich in 6 Kategorien einteilen lassen. Die erste Kategorie ist die Ansammlung der Daten, also dass gesammelte und zu einem Gesamtbild zusammengefügte Daten viel mehr über einen Nutzer aussagen können, als er dies bei dem Veröffentlichen einer einzelnen Information erwartet. So wird als Beispiel angebracht, dass eine Suche nach "French Mountains" alleine nicht sehr aussagekräftig ist, wenn aber kurz darauf noch nach "ski vacation" und "gift to grandchild" und weiterem gesucht wird wird schnell klar worum es sich bei der ersten Frage gehandelt hat. Wenn danach zum Beispiel noch nach "disabled access" gesucht wird entwickelt sich ein sehr umfangreiches Bild über die Familie des Suchmaschinen-Nutzers. Der zweite Punkt ist die Verzerrung, das heißt, dass Suchanfragen, wenn sie ohne Kontext gestellt sind, sehr schnell ein falsches Bild liefern können. Als Beispiel wird hier die Suchanfrage "assassinate US president" gezeigt - Behörden könnten hierbei schnell aufmerksam auf den Suchenden werden, obwohl sich jemand nur über die Geschichte von früheren Präsidenten informieren wollte. Seine dritte Kategorie ist der Ausschluss. So dürfen sich Nutzer in Europa zwar ihre bei Google gespeicherten Informationen ausgeben lassen, da darüber aber nur wenige Menschen Bescheid wissen gilt Googles Datenbank letzten Endes als "secret Database". Der vierte Punkt ist die "zweite Nutzung". So wird durch die Verwendung von Google der Datenschutzrichtlinie zugestimmt, die es Google erlaubt die Daten für weit mehr als der direkten Verwendung zu nutzen. Die letzten 2 Kategorien sind der Vertrauensbruch und die restlichen Probleme. So vertraut der Nutzer zum Beispiel darauf, dass Google ihre Daten vertraulich behandelt - und empfindet es als Vertrauensbruch wenn sie erfahren dass Google die Daten auch für andere Zwecke verwendet.

Als Analogon zu einer ähnlich tiefen Verbindung werden in" Facebook and online privacy: Attitudes, behaviors, and unintended consequences" \cite{debatin2009facebook} die Problematiken der Privatsphäre bei der Benutzung von Facebook untersucht. Genau wie auch Google ist Facebook ein Tool geworden das alltäglich von vielen Nutzern verwendet wird und dabei sehr tief in die Privatsphäre eingreifen kann. Wie auch in dieser Arbeit geht es bei der Untersuchung zu einem großen Teil um die Verhaltensweisen der Nutzer und darum wie viel sie über die Konsequenzen ihres Verhaltensmusters wissen. So wird zum Beispiel gefragt wie viele Nutzer der Meinung sind sich mit den Privatsphäreeinstellungen in Facebook auszukennen und ob sie sich in der Lage fühlen ihr Profil zumindest vor dem Zugriff Dritter zu schützen. Ähnliche Fragen werden auch in dieser Arbeit behandelt.

Es gibt vor allem im Bereich Social Network noch weitere Arbeiten die ähnliche Untersuchungen unternehmen, allerdings werden diese hier nicht dargestellt.