%%%%%%%%%%%%%%%%%%%%%%%%%%%%%%%%%%%%%%%%%%%%%%%%%%%%%%%%%%%%%%%%%%%%%%%%%%%%%%%
% 
% Prerequisites 
% Copyright (c) 2010 by tilo.mueller@rwth-aachen.de
% 
%%%%%%%%%%%%%%%%%%%%%%%%%%%%%%%%%%%%%%%%%%%%%%%%%%%%%%%%%%%%%%%%%%%%%%%%%%%%%%%

\chapter{Verwandte Arbeiten}
Die Problematik der Privatsphäre bei der Kommunikation mit Google Diensten ist ein Thema zu der es bereits einige Artikel und Forschungsarbeiten gibt. So beschreibt zum Beispiel \citet{tene2007google}, dass die größten Probleme der Privatsphäre sich in 6 Kategorien einteilen lassen.

Die erste Kategorie ist die Ansammlung der Daten, also dass gesammelte und zu einem Gesamtbild zusammengefügte Daten viel mehr über einen Nutzer aussagen können, als er dies bei dem Veröffentlichen einer einzelnen Information erwartet. So wird als Beispiel angebracht, dass eine Suche nach \glqq French Mountains\grqq\ alleine nicht sehr aussagekräftig ist, wenn aber kurz darauf noch nach \glqq ski vacation\grqq\ und \glqq gift to grandchild\grqq\ und weiterem gesucht wird wird schnell klar worum es sich bei der ersten Frage gehandelt hat. Wenn danach zum Beispiel noch nach \glqq disabled access\grqq\ gesucht wird entwickelt sich ein sehr umfangreiches Bild über die Familie des Suchmaschinen-Nutzers. 

Der zweite Punkt ist die Verzerrung, das heißt, dass Suchanfragen, wenn sie ohne Kontext gestellt sind, sehr schnell ein falsches Bild liefern können. Als Beispiel wird hier die Suchanfrage \glqq assassinate US president\grqq\ gezeigt - Behörden könnten hierbei schnell aufmerksam auf den Suchenden werden, obwohl sich jemand nur über die Geschichte von früheren Präsidenten informieren wollte.

Seine dritte Kategorie ist der Ausschluss. So dürfen sich Nutzer in Europa zwar ihre bei Google gespeicherten Informationen ausgeben lassen, da darüber aber nur wenige Menschen Bescheid wissen gilt Googles Datenbank letzten Endes als \glqq secret Database\grqq\ . 

Der vierte Punkt ist die \glqq sekundäre Nutzung\grqq . So wird durch die Verwendung von Google der Datenschutzrichtlinie zugestimmt, die es Google erlaubt die Daten für weit mehr als der direkten Verwendung zu nutzen. Als Beispiel für eine sekundäre Nutzung kann die Werbung gesehen weren. 

Die letzten 2 Kategorien sind der Vertrauensbruch und die restlichen Probleme. In der Arbeit wird William Prosser zitiert, welcher sagt, dass der Nutzer durch das Veröffentlichen seiner privaten Daten eine Verletzung seines Vertrauens sieht. Unter den restlichen Problemen wird unter anderem die Überwachung genannt.

Als Analogon zu einer ähnlich tiefen Verbindung werden in \glqq Facebook and online privacy: Attitudes, behaviors, and unintended consequences\grqq\ \citet{debatin2009facebook} die Problematiken der Privatsphäre bei der Benutzung von Facebook untersucht. Genau wie auch Google ist Facebook ein Tool geworden das alltäglich von vielen Nutzern verwendet wird und dabei sehr tief in die Privatsphäre eingreifen kann. Wie auch in dieser Arbeit geht es bei der Untersuchung zu einem großen Teil um die Verhaltensweisen der Nutzer und darum wie viel sie über die Konsequenzen ihres Verhaltensmusters wissen. So wird zum Beispiel gefragt wie viele Nutzer der Meinung sind sich mit den Privatsphäreeinstellungen in Facebook auszukennen. Außerdem werden sie gefragt, ob sie sich in der Lage fühlen ihr Profil zumindest vor dem Zugriff Dritter zu schützen. Ähnliche Fragen werden auch in dieser Arbeit behandelt.

Eine weitere verwandte Arbeit ist \glqq Media coverage of online social network privacy issues in Germany: A thematic analysis\grqq\ \citet{rizk2009media}. In diesem Paper wurde ein Teil der deutschen Medien auf das Thema \glqq Privatsphäre in sozialen Netzwerken\grqq\ durchsucht und die gesammelten Artikel wurden ausgewertet. Eines der Ergebnisse dieser Arbeit ist, dass es eine Beziehung zwischen Artikeln in den Medien und Änderungen in den Nutzungsbedingungen im Bereich Privatsphäre gibt. Das Paper kommt zu dem Schluss, dass diese Änderungen immer gleichzeitig oder kurz nach großen Mengen an Artikeln zu dem Thema kommen.

Ein Begriff der in den Ergebnissen dieser Arbeit immer wieder aufgetaucht ist ist \glqq Personalisierte Werbung\grqq . Dieser Begriff wird auch in dieser Arbeit behandelt.

Auch in \citet{Preibusch2007Ubiquitous} wird das Problem Privatsphäre in Sozialen Netzwerken behandelt. Der Artikel behandelt das Thema nicht nur im Bezug auf einzelne Personen sondern auch im Bezug auf Privatsphäre im Netzwerk. Insbesondere geht es hier darum, dass Person A die persönlichen Daten von Person B missbrauchen kann, solange sie diese Person im Sozialen Netzwerk kennt. Diese Arbeit unterteilt dazu Vertraulichkeit in 4 Teile, Private Daten, Gruppen-Daten, Daten für das Netzwerk und öffentliche Daten. Sämtliche Daten ab \glqq Gruppendaten \grqq\ sind für Person A ersichtlich. Wenn diese Person nun bösartig ist, gibt es einen Konflikt mit der Privatsphäre. 

Um dieses Problem zu lösen schlagen die Autoren eine Erweiterung des P3P (Platform for Privacy Preferences) vor. Mit dieser Erweiterung soll jeder Nutzer selbst festlegen können, wie sich welche Freundschaften zu anderen Nutzern auswirken. Zum Beispiel kann ein Nutzer private und öffentliche Freundschaften anlegen.


Eine weitere Arbeit mit ähnlichem Inhalt ist \citet{gross2005information}. Diese Arbeit listet einige Datenschutzauswirkungen auf, unter anderem werden \glqq Stalking\grqq , \glqq Re-Identifikation\grqq\ und der \glqq Fragile Datenschutz\grqq\ genannt.


Weitere verwandte Arbeiten sind \citet{hintz2014AGB}, \citet{mahmood2011privacy} und \citet{hintzPhishing}.

Es gibt noch weitere Arbeiten in diesem Bereich, diese werden aber nicht aufgelistet.